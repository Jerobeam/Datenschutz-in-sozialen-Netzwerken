\chapter{Einleitung}
Im Zeitalter des Internets gehören soziale Netzwerke wie \textit{Facebook} oder \textit{Twitter} zum Alltag. Um diese Netzwerke nutzen zu können, müssen die Anwender mit den Datenschutzbestimmungen der Dienstanbieter einverstanden sein. Dies schließt die Datenspeicherung der personenbezogenen Daten der Benutzer ein, in vielen Fällen sogar die Weitergabe der Daten an Dritte. Das Bundesdatenschutzgesetz und das Telemediengesetz regeln dabei die Persönlichkeitsrechte beim Umgang mit personenbezogenen Daten in Deutschland und bilden den Rahmen für die Erhebung, Verarbeitung und Nutzung dieser. Es bestimmt auch unter welchen Bedingungen soziale Netzwerke in Deutschland erhobene Daten in ein anderes Land transferieren können. Die wichtigsten Aspekte der beiden Gesetze werden in dieser Arbeit erläutert.\\
\\Im ersten Teil der Arbeit werden die Grundlagen des Bundesdatenschutzgesetzes dargelegt. Dabei werden u.a. auf den Hintergrund des Gesetzes, den Zweck und die Regelungen der Zulässigkeit der Datenerhebung eingegangen. Anschließend folgt die Erläuterung der wichtigsten Aspekte des Telemediengesetzes, um darauffolgend die deutschen Gesetzesregelungen auf internationale Datenschutzregelungen zu beziehen. Die Arbeit schließt mit einem Fazit.