\chapter{Einleitung}
Im Zeitalter des Internets gehören soziale Netzwerke wie \textit{Facebook} oder \textit{Twitter} zum Alltag. Um diese Netzwerke nutzen zu können, müssen die Anwender mit den Datenschutzbestimmungen der Diensteanbieter einverstanden sein. Dies schließt die Datenspeicherung der personenbezogener Daten der Benutzer ein, in vielen Fällen sogar die Weitergabe der Daten an Dritte. Das Bundesdatenschutzgesetz und das Telemediengesetz regeln dabei die Persönlichkeitsrechte beim Umgang mit personenbezogenen Daten und bilden den Rahmen für die Erhebung, Verarbeitung und Nutzung dieser. Die wichtigsten Aspekte der beiden Gesetze werden in dieser Arbeit erläutert.\\
\\Im ersten Teil der Arbeit werden die Grundlagen des Bundesdatenschutzgesetzes dargelegt. Dabei wird u.a. auf den Hintergrund des Gesetzes, den Zweck und die Regelungen der Zulässigkeit der Datenerhebung. Anschließend folgt die Erläuterung der wichtigsten Aspekte des Telemediengesetzes, um anschließend die deutschen Gesetzesregelungen auf internationale Datenschutzregelungen zu beziehen. Die Arbeit schließt mit einem Fazit.
\chapter{Grundlagen des Bundesdatenschutzgesetzes}
Im nachfolgenden Kapitel werden die grundlegenden Aspekte des Bundesdatenschutzgesetzes vermittelt, welche zum allgemeinen Verständnis dieser Arbeit notwendig sind. Im ersten Teil wird der Hintergrund des Gesetzes näher erläutert, um im zweiten Teil die wichtigsten Texte zu erläutern.
\section{Hintergrund des Gesetzes}
Die amtliche Anmerkung zum Bundesdatenschutzgesetz (\acs{BDSG}) beschreibt den allgemeinen Hintergrund des Gesetzes wie folgt:\\
\\\glqq Dieses Gesetz dient der Umsetzung der Richtlinie 95/46/EG des Europäischen Parlaments und des Rates vom 24. Oktober 1995 zum Schutz natürlicher Personen bei der Verarbeitung personenbezogener Daten und zum freien Datenverkehr (ABl. EG Nr. L 281 S. 31).\grqq \autocite[][]{DeJureBDSG} \\
\\Die Richtlinie 95/46/EG des Europäischen Parlaments und des Rates vom 24. Oktober 1995 beordert den Schutz natürlicher Personen bei der Verarbeitung personenbezogener Daten. Die Gründe für das Verfassen der Richtlinie sind u.a. einerseits die Notwendigkeit der Übermittlung von personenbezogenen Daten von einem Mitgliedsstaat der Europäischen Union (\acs{EU}) in den anderen zur Errichtung eines funktionierenden Binnenmarktes. Bei dieser Übermittlung sind jedoch die Grundrechte der Personen zu wahren. Andererseits gelten in den Mitgliedsstaaten unterschiedliche Schutzniveaus der Personenfreiheiten und -rechte bei der Verarbeitung personenbezogener Daten, was die Übermittlung dieser Daten zwischen den verschiedenen Staaten verhindern kann. So könnten zahlreiche gemeinsame Wirtschaftsaktivitäten gehemmt oder der Wettbewerb verfälscht werden. Aus diesem Grund ist ein gleichwertiges Schutzniveau zur Beseitigung der Hemmnisse unerlässlich. In der Gemeinschaft ist die Angleichung der Rechtsvorschriften erforderlich.\autocite[vgl.][]{EU.1995}\\
\\
\chapter{Grundlagen des Telemediengesetzes im Bezug auf soziale Netzwerke}
Das BDSG regelt den Datenschutz bei der allgemeinen Erhebung von personenbezogenen Daten, jedoch nicht explizit für die Datenerhebung bei der Benutzung von Telemedien durch Diensteanbieter, darunter auch soziale Netzwerke. Für diesen Zweck ist 2007 das Telemediengesetz (\acs{TMG}) in Kraft getreten.\autocite[vgl.][]{klicksafe.de} \autocite[vgl.][]{shr.de} In diesem Kapitel werden die Grundlagen des TMGs vermittelt und im Kontext der Anwendung von sozialen Netzwerken behandelt.
\section{Begriffsbestimmungen und Pflichten}
Laut \S 2 TMG ist ein Diensteanbieter \glqq jede natürliche oder juristische Person, die eigene oder fremde Telemedien zur Nutzung bereithält oder den Zugang zur Nutzung vermittelt (...).\grqq \ Weiterhin ist ein Nutzer laut \S 2 TMG jede natürlich oder juristische Person, die Telemedien nutzt, um insbesondere Informationen zu erhalten. Abschließend werden Telemedien als Verteildienste bezeichnet, welche laut \S 2 TMG \glqq im Wege einer Übertragung von Daten ohne individuelle Anforderung gleichzeitig für eine unbegrenzte Anzahl von Nutzern erbracht werden.\grqq \ Dazu gehören nahezu alle Angebote im Internet, wie Shopping-Portale, Online-Dienste (wie z.B. Wetter- oder Nachrichtenauskünfte) oder Suchmaschinen.\autocite[vgl.][]{shr.de}\\
\\Wann und unter welchen Bedingungen personenbezogene Daten nun aber tatsächlich erhoben werden dürfen, regeln \S\S 12, 13 TMG.  \S 12 Abs. 1 TMG besagt, dass Diensteanbieter personenbezogene Daten nur erheben und verwenden dürfen, wenn das TMG oder eine andere Telemedien-bezügliche Rechtsvorschrift es erlaubt oder der Nutzer eingewilligt hat. Gleiches gilt für die Verwendung der Daten für andere Zwecke. Zu Beginn des Nutzungsvorgangs hat der Diensteanbieter den Betroffenen über die Erhebung und Verwendung der Daten laut \S 13 Abs. 1 TMG zu unterrichten. Die Einwilligung des Nutzers kann dabei elektronisch eingeholt werden, sofern der Diensteanbieter sicherstellt, dass die Einwilligung bewusst und eindeutig erteilt wird, die Einwilligung protokolliert wird, der Inhalt der Einwilligung jederzeit abrufbar ist und der Nutzer die Einwilligung jederzeit widerrufen kann (\S 13 Abs. 2 TMG).
\section{Speicherung von Bestands- und Nutzungsdaten}
\S 14 Abs. 1 TMG spezifiziert eine Art der Daten, welche von den Diensteanbietern erhoben und verwendet werden dürfen: die Bestandsdaten. Bestandsdaten gelten als Daten, welche \glqq für die Begründung, inhaltliche Ausgestaltung oder Änderung eines Vertragsverhältnisses zwischen dem Diensteanbieter und dem Nutzer über die Nutzung von Telemedien erforderlich sind (Bestandsdaten).\grqq \ Wenn eine zuständige Stelle eine Auskunft über Bestandsdaten anordnet, sofern dies staatliche Zwecke, wie z.B. der Strafverfolgung, notwendig ist, dürfen Diensteanbieter diese Auskunft erteilen.\\
\\Neben den Bestandsdaten sieht das TMG ebenfalls die Erhebung und Nutzung von Nutzungsdaten vor. Laut \S 15 Abs. 1 TMG sind Nutzungsdaten erforderlich, um die Inanspruchnahme von Telemedien zu ermöglichen und abzurechnen. Dabei sind Nutzungsdaten insbesondere Merkmale zur Identifikation des Nutzers, Angaben zum Beginn und Ende sowie zum Umfang der jeweiligen Nutzung und Angaben über die vom Nutzer ins Anspruch genommenen Telemedien. \S 15 Abs. 3 erlaubt den Diensteanbietern die Nutzung der Daten zur Erstellung von Nutzungsprofilen für Zwecke der Werbung, Marktforschung oder zur bedarfsgerechten Gestaltung der Telemedien führen, wenn der Nutzer dem nicht widerspricht. Weiterhin darf der Diensteanbieter die Daten auch nach dem Ende des Nutzungsvorgangs für Zwecke der Abrechnung verwenden und diese für demselben Zweck an andere Dienstanbieter und Dritte weitergeben. Zur Marktforschung anderer Diensteanbieter dürfen die Daten anonymisiert übermittelt werden (\S 15 Abs. 4,5 TMG).\\
\section{Die Gesetze zum Datenschutz im Kontext sozialer Netzwerke}
In der Praxis sind die Diensteanbieter von sozialen Netzwerken darauf angewiesen, mit ihrem Angebot Profit zu erzielen. Dabei werden die verwendeten Daten, wie weiter oben erläutert, häufig anonymisiert an Dritte verkauft und weitergegeben, um die grundsätzliche kostenlose Nutzung ihrer Dienste auszugleichen. Der Verkauf der Daten wird über die Allgemeinen Geschäftsbedingungen (\acs{AGB}) abgesichert, welche vor der Nutzung vom Benutzer durchgelesen werden sollten und bestätigt werden. Somit ist die Unterrichtungspflicht der Diensteanbieter gegenüber den Nutzern erfüllt. Jedoch ist nicht alles, was die Anbieter in den AGB vermerken, automatisch rechtskräftig. Aus diesem Grund geht der Bundesverband der Verbraucherzentralen gegen einzelne, für Nutzer besonders nachteilige Klauseln vor.\autocite[vgl.][]{klicksafe.de2}\\
\\Im Jahr 2015 beispielsweise hat der Verbraucherschutz \textit{Facebook} abgemahnt, nachdem der Anbieter des sozialen Netzwerks seine AGBs aktualisiert hat und diese aus Sicht des Verbands gegen 19 Klauseln des deutschen Rechts verstoßen. Das Geschäftsmodell von Facebook basiert auf dem Motto \textit{Facebook ist und bleibt kostenlos}, jedoch wurden mit der Einführung der aktualisierten AGB personenbezogene Nutzerdaten an werbetreibende Unternehmen weiterverkauft. Die Verbraucherzentrale argumentiert, dass Facebook sein Geschäftsmodell verharmlosen und Transparenz verhindern würde. Trotz Mahnung der der Verbraucherzentrale konnte noch keine Einigung mit Facebook erzielt werden, sodass den Nutzern aktuell (Stand April 2017) nichts anderes übrig bleibt, als das soziale Netzwerk zu verlassen.\autocite[vgl.][]{SpiegelOnline} \autocite[vgl.][]{Verbraucherzentrale}\\
\\Weiterhin gilt ebenfalls für soziale Netzwerke das Prinzip der Datensparsamkeit. Dies bedeutet, dass auch soziale Netzwerke die Datenmenge, die der Nutzer dem Netzwerk übermittelt, auf ein Minimum zu reduzieren hat. Außerdem ist es in einzelnen Fällen für den Nutzer möglich, sich unter einem Pseudonym anzumelden und selektiv mit den Daten umzugehen: \glqq bei den wenigsten Sozialen Netzwerken ist es wirklich notwendig, seinen vollen Namen, die echte Adresse oder die Telefonnummer anzugeben, um den Dienst nutzen zu können. Schließlich kauft man dort (...) in der Regel nicht ein oder erhält Rechnungen, wofür der Anbieter Geschäftsdaten benötigen würde.\grqq \autocite[][]{klicksafe.de2} \autocite[vgl.][]{klicksafe.de2}\\
\\An dieser Stelle sind die nötigen Grundlagen zum BDSG und TMG vermittelt. Im folgenden Kapitel werden internationale Datenschutzregelungen im Kontext der Verwendung von sozialen Netzwerken näher beleuchtet.

\chapter{Umgang mit personenbezogenen Daten anhand von Praxisbeispielen}
Das vorliegende Kapitel befasst sich damit, welche personenbezogenen Daten die drei sozialen Netzwerke WhatsApp, Facebook und Twitter von ihren Nutzern für welche Zwecke speichern, und inwiefern der Nutzer Zugang und Gewalt über die zu ihm gespeicherten Daten besitzt. Dazu werden Teilaspekte der AGBs dieser Betreiber genauer durchleuchtet. Wie die Statistik in \vref{fig:ranking} zeigt, besitzen die ausgewählten Netzwerke eine sehr hohe Reichweite und gehören zu denjenigen sozialen Netzwerken, welche heutzutage am meisten verwendet werden.

\section{WhatsApp}
In den Nutzungsbestimmungen von WhatsApp steht zu Anfang geschrieben, dass der Nutzer durch das Verwenden der Dienste von WhatsApp eine mehr oder weniger allumfassende Lizenz zur "`Nutzung, Reproduktion, Verbreitung, Erstellung abgeleiteter Werke, Darstellung und Aufführung der Informationen, die [der Nutzer hochlädt, übermittelt, speichert, sendet oder empfängt]"' gewährt \autocite[][]{WhatsAppInc..2017}. Zwar beschränkt sich WhatsApp in dieser Hinsicht darauf, diese Lizenz lediglich für die Dienstbetreibung, z.B. für die Nachrichtenübermittlung, zu nutzen. Nichtsdestotrotz wäre es möglich, diese Lizenz auch für weitere Marktforschungszwecke, Trendanalysen o.Ä. zu verwenden, was von WhatsApp jedoch nicht erwähnt wird. Zu den übermittelten Informationen zählen die eigene Handynummer, die Kontakte aus dem Adressbuch, aufgerufene Webseiten aus dem Messenger heraus und die Inhalte, die über die Teilen-Funktion von WhatsApp versendet werden. Darüber hinaus erhält WhatsApp Standorte, die der Nutzer selber teilt, oder die von anderen Nutzern mitgeteilt bekommen. Auch sein Online-Status und Zuletzt-Online-Status wird gespeichert \autocite[vgl.][]{WhatsAppInc..2017}.
\par
Damit WhatsApp seine Dienste bereitstellen und verbessern kann, erhebt es nicht nur die oben genannten Informationen, sondern teilt sie auch mit der Facebook-Unternehmensgruppe und Drittanbietern \autocite[vgl.][]{WhatsAppInc..2017}.
\par
Der Nutzer hat die Möglichkeit, seinen WhatsApp-Account zu löschen. Dies garantiert jedoch nicht, dass WhatsApp alle gespeicherten personenbezogenen Daten dieses Nutzers tatsächlich löscht, da WhatsApp diejenigen Daten löscht, die nach der Löschung des Accounts nicht mehr zum Betreiben und Bereitstellen seiner Dienste benötigt werden \autocite[vgl.][]{WhatsAppInc..2017}.

\section{Facebook}
Im Grunde speichert Facebook die Daten zu allen Informationen und allen Aktivitäten, die mit einem Nutzer zu tun haben. Dazu gehören erstellte und geteilte Posts, Wohnort, Geburtstag, angeschaute Inhalte, häufige Kontaktpersonen, Gerätestandorte, Namen des Mobilfunk- und Internetdienstanbieter, Besuche von Webseiten, die den "`Gefällt-mir"'-Button eingebettet haben und vieles mehr \autocite[vgl.][]{FacebookInc..2017}.
\par
Facebook nutzt diese gesammelten Informationen um personalisierte Werbung und Seiten- oder Freundesvorschläge anzuzeigen (s. \vref{KontextSozialeNetzwerke}) . Auch nutzt es sie um Personen in hochgeladenen Bildern zu erkennen. Auch Facebook gibt gesammelte Informationen an die Facebook-Unternehmensgruppe und an Dritte weiter, um z.B. die Wirksamkeit von Werbeanzeigen und Diensten zu messen \autocite[vgl.][]{FacebookInc..2017}. 
\par
Im Gegensatz zu WhatsApp hingegen ist die Auskunft über die gesammelten Daten der eigenen Person transparenter gestaltet. Der Nutzer hat nämlich die Möglichkeit über einen speziellen Facebook-Dienst eine Datei herunterzuladen, die Auskunft über alle personenbezogenen Daten gibt, die über den Nutzer gesammelt wurden. Sie enthält z.B. seine Adresse, seine Klicks auf Werbeanzeigen, hinzugefügte Apps über Facebook, aktive Sitzungen, Gefällt-mir"'-Angaben, Fotos, Beiträge, gelöschte Freunde, und in Facebook getätigte Suchanfragen \autocite[vgl.][]{FacebookInc..2017b}.

\section{Twitter}
Das letzte Beispiel stellt das soziale Netzwerk Twitter dar, welches im Vergleich zu den anderen beiden Netzen kein Bestandteil der Facebook-Unternehmensgruppe ist. Twitter speichert auf der einen Seite grundlegende Accountinformationen wie E-Mail-Adresse, Geburtsdatum, Kurzbeschreibung, Bild und Telefonnummer der Nutzers.  Auf der anderen Seite werden Daten zu Tweets erhoben, die der Nutzer verfasst, als "`gefällt-mir"' markiert, oder teilt. Zusätzlich kann der Nutzer optional sein Adressbuch hochladen, um Follower-Vorschläge zu erhalten. In diesem Fall werden diese Adressbuchdaten ebenfalls von Twitter gespeichert. Darüber hinaus behält sich Twitter vor, Standortdaten zu sammeln, wenn diese vom Nutzer geteilt werden \autocite[vgl.][]{TwitterInc..2017}.
\par
Twitter benutzt diese Daten zur Bereitstellung, Bewertung und Verbesserung der eigenen Dienste, und um dem Nutzer personenbezogenen und standortabhängige Werbung, Trends, Geschichten und Follower-Vorschläge anzuzeigen \autocite[vgl.][]{TwitterInc..2017}. 
\par
Nicht-öffentliche, personenbezogene Daten gibt Twitter nur nach konkreten Anweisungen der Nutzer an Dritte weiter, wenn der Nutzer z.B. einer anderen Software den Zugriff auf den eigenen Twitter-Account gewährt. Twitter Nutzungsbedingung sieht auch vor, dass jeder Nutzer die über sich gespeicherten, personenbezogenen Daten ansehen, anpassen oder löschen kann \autocite[vgl.][]{TwitterInc..2017}.

\chapter{Internationale Datenschutzregelungen}
\section{Anwendbares Recht}
Das deutsche \ac{BDSG} wendet grundsätzlich das Territorialprinzip an. Demzufolge müssen ausländische Stellen bzw. ausländische Betreiber von sozialen Netzwerken deutsches Recht berücksichtigen, wenn diese in Deutschland personenbezogene Daten erheben wollen \autocite[vgl.][]{ICS.2011}.
\par
Eine Ausnahme existiert für Stellen, die ihren Sitz in einem Mitgliedsstaat der \ac{EU} haben und zusätzlich keinen Sitz in Deutschland haben. Denn dann wird gemäß Art. 25 der EU-Datenschutzrichtlinie 95/46/EG das Sitzlandprinzip angewandt, wenn personenbezogene Daten in Deutschland erhoben werden. Das Sitzlandprinzip besagt, dass das Recht zur Datenerhebung und zum Datentransfer desjenigen Landes anzuwenden ist, in dem die datenerhebende Stelle ihren Sitz hat \autocite[vgl.][]{ICS.2011} \autocite[vgl.][]{EG.1995}.

\section{Prüfungsstufen zur Datenübermittlung ins Ausland}
Damit eine Stelle bzw. ein soziales Netzwerk, welches Daten in den Deutschland erhebt, diese erhobenen Daten in ein anderes Land transferieren darf, gilt es zwei Prüfungsstufen zu bewältigen, welche in den folgenden Unterkapiteln genauer erläutert werden.

\subsection{Prüfungsstufe 1: Zulässigkeit der Datenübermittlung}
In der ersten Prüfungsstufe muss sich die erhebende Stelle bzw. das soziales Netzwerk vergewissern, ob es überhaupt befugt ist, die erhobenen Daten aus Deutschland heraus in ein anderes Land zu transferieren. Diese Befugnis kann entweder auf einem Gesetz oder auf der Einwilligung der betroffenen Person, deren erhobenen Daten ins Ausland übermittelt werden sollen, gemäß §4a BDSG beruhen \autocite[vgl.][]{LDI.2017}. Letzteres spielt für soziale Netzwerke eine besondere Rolle. Schließlich müssen Nutzer von sozialen Netzwerken ihren Datenschutzrichtlinien und damit dem Datentransfer ins Ausland bereits bei der Registrierung zustimmen.
\par
Zusätzlich ist die erhebende Stelle den Grundprinzipien der Datenspeicherung verpflichtet, welche in \vref{Grundlagen} gelistet und erläutert sind \autocite[vgl.][]{LDI.2017}.

\subsection{Prüfungsstufe 2: Prüfung des Datenschutzniveaus im Empfängerstaat}
Bei der zweiten Prüfungsstufe wird das Datenschutzniveau im Empfängerstaat untersucht. Hierzu wird eine Fallunterscheidung vorgenommen.

\subsubsection{Fall 1: Transfer innerhalb der EU}
Da gemäß §4b Abs. 1 BDSG ein hohes Datenschutzniveau in allen Mitgliedstaat der \acl{EU} vorliegt, kann die Datenübermittlung zu einer Empfängerstelle innerhalb der \acl{EU} ohne weitere Vorkehrungen direkt folgen. Neben den Mitgliedsstaaten der \ac{EU} wird auch Island, Norwegen und Liechtenstein ein hohes Datenschutzniveau zugeschrieben, weshalb der Datentransfer in diese Länder ebenfalls unproblematisch ist \autocite[vgl.][]{LDI.2017}.

\subsubsection{Fall 2: Transfer in Drittstaaten mit Angemessenheitsentscheidung}
Als "`Drittstaaten"' oder "`Drittländer"' werden nach Art. 25 der EU-Datenschutzrichtlinie 95/46/EG Empfängerstaaten bezeichnet, welche ihren Sitz nicht in der EU haben\autocite[vgl.][]{LDI.2017} \autocite[vgl.][]{EG.1995}.
\par
Manche dieser Drittstaaten haben im Sinne der \ac{EU} ein angemessenes Datenschutzniveau, weshalb diesen Drittstaaten gemäß Art. 25 Abs. 6 der EU-Datenschutzrichtlinie 95/46/EG eine sogenannte Angemessenheitsentscheidung ausgesprochen wurde. Stand September 2016 trifft dies für die folgenden Staaten zu: Andorra, Argentinien, Kanada, Schweiz, Färöer-Inseln, Guernsey, Israel, Isle of Man, Jersey, Neuseeland und Uruguay. Liegt die Empfängerstelle, zu der der Betreiber des sozialen Netzwerkes von Deutschland senden will, in einem dieser Länder, so kann dies ohne Weiteres getan werden \autocite[vgl.][]{LDI.2017} \autocite[vgl.][]{EG.1995}.

\subsubsection{Fall 3: Transfer in die USA}
Ehemalig regelte die 2000 getroffene Übereinkunft, das sogenannte "`Safe Habor"'-Abkommen, den Datentransfer von Deutschland in die USA. Es besagte, dass Unternehmen in den USA ein angemessenes Datenschutzniveau im Sinne des Artikel 25 Absatz 6 der Datenschutzrichtlinie 95/46 gewährleisten, wenn sie sich den Prinzipien des Safe Habor-Abkommen per Selbstzertifizierung verpflichten \autocite[vgl.][]{BDFI.2017}. Diese Prinzipien sind die "`sieben "Grundsätze des ,sicheren Hafens‘ zum Datenschutz"  (Informationspflicht, Wahlmöglichkeit, Weitergabe, Sicherheit, Datenintegrität, Auskunftsrecht und Durchsetzung)"' \autocite[][]{BDFI.2017}. Mit dem Urteil zur Rechtssache C-362/14, welches auch das "`Schrems-Urteil"' genannt wird, wurde das "`Safe Habor"'-Abkommen am 06.10.2015 für ungültig erklärt \autocite[vgl.][]{BDFI.2017}. Zu dem besagten Schrems-Urteil kam der Europäische Gerichtshof, da der österreichische Jurist Maximilian Schrems darüber empört war, was das soziale Netzwerk Facebook alles über ihn gespeichert hatte und deshalb gegen Facebook vorgegangen ist. Denn unter anderen hatte Facebook Daten von ihm gespeichert, die Schrems für gelöscht gehalten hatte \autocite[vgl.][]{Welt.2015}. Trotz der Aufhebung des Urteils wird vom US-Handelsministerium weiterhin eine Liste mit den Safe Habor-zertifizierten Unternehmen geführt, da die Prinzipien für die Daten, die von den Unternehmen unter der dem Safe Habor-Abkommmen gespeichert wurden, gelten bis sie gelöscht werden \autocite[vgl.][]{BDFI.2017}.
\par
Als Nachfolger von Safe Habor gibt es seit 12.07.2016 das "`Privacy Shield"'-Abkommen zwischen der \ac{EU} und USA. Dieses Abkommen definiert Regelungen, die für ein angemessenes Datenschutzniveau einzuhalten sind. Amerikanische Unternehmen haben die Möglichkeit, sich zur Einhaltung der Privacy Shield-Regelungen zu verpflichten, was die Datenübermittlung von Deutschland nach USA erlaubt \autocite[vgl.][]{LDI.2017}. Das weitverbreitete soziale Netzwerk Twitter hat laut ihren Datenschutzrichtlinien den Regelungen des Privacy Shield-Abkommen verpflichtet, was Twitter den Transfer von deutschen Daten in die USA ermöglicht \autocite[vgl.][]{TwitterInc..2017}.

\subsubsection{Fall 4: Sonstige Drittstaaten}
Will ein soziales Netzwerk personenbezogene Daten von Deutschland in ein Drittland übermitteln, welches sich keinem der drei zuvor gelisteten Fälle zuordnen lässt, dann muss die übermittelnde Stelle, also der Betreiber des sozialen Netzwerkes, gemäß §4b Abs. 3 und 5 BDSG überprüfen, ob ein angemessenes Datenschutzniveau in bei der Empfängerstelle vorliegt \autocite[vgl.][]{LDI.2017}. Ist diese Überprüfung erfolgreich, so ist die Datenübermittlung legitim. Wenn nicht, dann ist ein Transfer zur Empfängerstelle nur dann möglich, wenn einer der sechs in §4c BDSG beschriebenen Tatbestände eintritt. Der erste Tatbestand in §4c Abs. 1 Nr. 1 BDSG ist derjenige, der als einziger für Betreiber von sozialen Netzwerken greifen kann. Dieser Tatbestand besagt nämlich, dass ein Datentransfer zu einer Empfängerstelle ohne angemessenem Datenschutzniveau dann stattfinden darf, wenn die betroffene Person dazu eingewilligt hat. Damit diese Einwilligung gültig ist, muss die betroffene Person davor ausdrücklich darüber informiert werden, dass ihre Daten ohne angemessenen Datenschutz außerhalb Deutschlands gespeichert oder verarbeitet werden \autocite[vgl.][]{LDI.2017}. Der Kurznachrichtendienst WhatsApp ist ein soziales Netzwerk, deren Datenexport auf dieser Einwilligung beruht. In ihren Datenschutzrichtlinien, die jeder Nutzer akzeptieren muss, heißt es: 
\begin{quotation}
	"`Du akzeptierst unsere Informationspraktiken, (...) sowie die Übertragung und Verarbeitung deiner Informationen in die/den USA und andere/n Länder/n weltweit, (...) und zwar unabhängig davon, wo du unsere Dienste nutzt. Du erkennst an, dass die Gesetze, Vorschriften und Standards des Landes, in dem deine Informationen gespeichert oder verarbeitet werden, von denen deines eigenen Landes abweichen können."' \autocite[][]{WhatsAppInc..2017}
\end{quotation}

\par
Greift keiner der in §4c BDSG beschriebenen Tatbestände, dann gibt es drei letzte Optionen, die ein Datentransfer zu einer Empfängerstelle in einem Drittstaat ohne angemessenen Datenschutzniveau legitim ist:
\begin{enumerate}
	\item Die Empfängerstelle kann einen EU-Standardvertrag unterzeichnen, welcher die Empfängerstelle dazu verpflichtet die Persönlichkeitsrechte der betroffenen Personen zu wahren \autocite[vgl.][]{LDI.2017}. Facebook gibt in seinen Datenschutzrichtlinien vor, EU-Standardverträge für den Datentransfer aus Europa heraus abgeschlossen zu haben \autocite[vgl.][]{FacebookInc..2017}.
	 
	\item Die Empfängerstelle kann einen Individualvertrag aufsetzen, welcher ebenfalls die Wahrung der Persönlichkeitsrechte der betroffenen garantiert \autocite[vgl.][]{LDI.2017}.
	
	\item Gehört die Empfängerstelle zu einem Konzern, so kann sich dieser Konzern einer verbindlichen Konzernregelung zum Datenschutz verpflichten, welche die Persönlichkeitsrechte der betroffenen Personen wahren. Dies ist besonders für global-aktiven Konzernen interessant, da solche verbindlichen Konzernregelung den Datentransfer zur allen Empfängerstellen des Konzern erlauben, unabhängig davon, in welchem Land sie ansässig sind \autocite[vgl.][]{LDI.2017}.
\end{enumerate}

\chapter{Fazit}
Für soziale Netzwerke sind insbesondere die deutschen und internationalen rechtlichen Regelungen zur Erhebung, Verarbeitung und Weiterleitung von personenbezogenen Daten ausschlaggebend. In Deutschland wird dies durch das Bundesdatenschutzgesetz geregelt.  Es schützt laut §1 Abs. 1 BDSG den Einzelnen davor, dass er durch den Umgang mit seinen personenbezogenen Daten in seinem Persönlichkeitsrecht beeinträchtigt wird. Betreiber von sozialen Netzwerken sind in Deutschland dazu verpflichtet, Erhebung, Verarbeitung und Nutzung personenbezogener Daten dem Prinzip der Datensparsamkeit zu unterstellen. Das heißt, dass sie so wenig Daten wie möglich speichern sollen. Des Weiteren sind sie gemäß der Unterrichtungspflicht aus §4 Abs. 3 BDSG dazu verpflichtet den Betroffen genauestens darüber zu informieren, welche Daten von ihm erhoben werden.
\par
Aber auch der rasche Wandel von Regelungen, Richtlinien, Abkommen und Gesetze im Kontext des Datenschutzes für Betreiber eines sozialen Netzwerkes zu berücksichtigen. Das jüngste Beispiel hierfür liefert das Schrems-Urteil vom 06.10.2015, in dem das Safe Habor-Abkommen zwischen den USA und Deutschland zum sicheren Datentransfer kurzerhand für ungültig erklärt wurde.
\par
Darüber hinaus spielt aber auch der Nutzer eine wichtige Rolle. Er möchte die Angebote eines sozialen Netzwerkes nutzen, kann mit diesem aber keine individuellen Nutzungsbedingungen aushandeln. Durch dieses strukturelle Ungleichgewicht kann ein Betreiber eines sozialen Netzwerkes im Rahmen des \ac{BDSG} diktieren, welche Daten er erhebt und was er damit macht. Das Musterbeispiel stellt das Netzwerk Facebook dar, welches soviel personenbezogene Daten erhebt, wie nur möglich. Aus diesem Grund sollte sich jeder Nutzer immer zwei Mal überlegen, welche Informationen er von sich preisgibt. Denn sie werden mit hoher Wahrscheinlich eine bleibende Spur im sozialen Netzwerk hinterlassen.