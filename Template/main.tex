\chapter{Inhalt hier hin packen}
Sample Cite \autocite[vgl.][]{Sample}

\chapter{Internationale Datenschutzregelungen}
\section{Anwendbares Recht}
Das deutsche \ac{BDSG} wendet grundsätzlich das Territorialprinzip an. Demzufolge müssen ausländische Stellen bzw. ausländische Betreiber von sozialen Netzwerken deutsches Recht berücksichtigen, wenn diese in Deutschland personenbezogene Daten erheben wollen \autocite[vgl.][]{ICS.2011}.
\par
Eine Ausnahme existiert für Stellen, die ihren Sitz in einem Mitgliedsstaat der \ac{EU} haben und zusätzlich keinen Sitz in Deutschland haben. Denn dann wird gemäß Art. 25 der EU-Datenschutzrichtlinie 95/46/EG das Sitzlandprinzip angewandt, wenn personenbezogene Daten in Deutschland erhoben werden. Das Sitzlandprinzip besagt, dass das Recht zur Datenerhebung und zum Datentransfer desjenigen Landes anzuwenden ist, in dem die datenerhebende Stelle ihren Sitz hat \autocite[vgl.][]{ICS.2011} \autocite[vgl.][]{EG.1995}.

\section{Prüfungsstufen zur Datenübermittlung ins Ausland}
Damit eine Stelle bzw. ein soziales Netzwerk, welches Daten in den Deutschland erhebt, diese erhobenen Daten in ein anderes Land transferieren darf, gilt es zwei Prüfungsstufen zu bewältigen, welche in den folgenden Unterkapiteln genauer erläutert werden.

\subsection{Prüfungsstufe 1: Zulässigkeit der Datenübermittlung}
In der ersten Prüfungsstufe muss sich die erhebende Stelle bzw. das soziales Netzwerk vergewissern, ob es überhaupt befugt ist, die erhobenen Daten aus Deutschland heraus in ein anderes Land zu transferieren. Diese Befugnis kann entweder auf einem Gesetz oder auf der Einwilligung der betroffenen Person, deren erhobenen Daten ins Ausland übermittelt werden sollen, gemäß §4a BDSG beruhen \autocite[vgl.][]{LDI.2017}. Letzteres spielt für soziale Netzwerke eine besondere Rolle. Schließlich müssen Nutzer von sozialen Netzwerken ihren Datenschutzrichtlinien und damit dem Datentransfer ins Ausland bereits bei der Registrierung zustimmen.
\par
Zusätzlich ist die erhebende Stelle den Grundprinzipien der Datenspeicherung verpflichtet, welche in \ref{Grundzüge} gelistet und erläutert sind \autocite[vgl.][]{LDI.2017}.

\subsection{Prüfungsstufe 2: Prüfung des Datenschutzniveaus im Empfängerstaat}
Bei der zweiten Prüfungsstufe wird das Datenschutzniveau im Empfängerstaat untersucht. Das Ergebnis
%TODO

\subsubsection{Fall 1: Transfer innerhalb der EU}
Da gemäß §4b Abs. 1 BDSG ein hohes Datenschutzniveau in allen Mitgliedstaat der \acl{EU} vorliegt, kann die Datenübermittlung zu einer Empfängerstelle innerhalb der \acl{EU} ohne weitere Vorkehrungen direkt folgen. Neben den Mitgliedsstaaten der \ac{EU} wird auch Island, Norwegen und Liechtenstein ein hohes Datenschutzniveau zugeschrieben, weshalb der Datentransfer in diese Länder ebenfalls unproblematisch ist \autocite[vgl.][]{LDI.2017}.

\subsubsection{Fall 2: Transfer in Drittstaaten mit Angemessenheitsentscheidung}
Als "`Drittstaaten"' oder "`Drittländer"' werden nach Art. 25 der EU-Datenschutzrichtlinie 95/46/EG Empfängerstaaten bezeichnet, welche ihren Sitz \autocite[vgl.][]{LDI.2017} \autocite[vgl.][]{EG.1995}.
\par
Manche dieser Drittstaaten haben im Sinne der \ac{EU} ein angemessenes Datenschutzniveau, weshalb diesen Drittstaaten gemäß Art. 25 Abs. 6 der EU-Datenschutzrichtlinie 95/46/EG eine sogenannte Angemessenheitsentscheidung ausgesprochen wurde. Stand September 2016 trifft dies für die folgenden Staaten zu: Andorra, Argentinien, Kanada, Schweiz, Färöer-Inseln, Guernsey, Israel, Isle of Man, Jersey, Neuseeland und Uruguay. Liegt die Empfängerstelle, zu der der Betreiber des sozialen Netzwerkes von Deutschland senden will, in einem dieser Länder, so kann dies ohne Weiteres getan werden \autocite[vgl.][]{LDI.2017} \autocite[vgl.][]{EG.1995}.

\subsubsection{Fall 3: Transfer in die USA}
Ehemalig regelte die 2000 getroffene Übereinkunft, das sogenannte "`Safe Habor"'-Abkommen, den Datentransfer von Deutschland nach USA. Es besagte, dass Unternehmen in den USA ein angemessenes Datenschutzniveau im Sinne des Artikel 25 Absatz 6 der Datenschutzrichtlinie 95/46 gewährleisten, wenn sie sich den Prinzipien des Safe-Habor-Abkommen per Selbstzertifizierung verpflichten \autocite[vgl.][]{BDFI.2017}. Diese Prinzipien sind die "`sieben "Grundsätze des ,sicheren Hafens‘ zum Datenschutz"  (Informationspflicht, Wahlmöglichkeit, Weitergabe, Sicherheit, Datenintegrität, Auskunftsrecht und Durchsetzung)"' \autocite[][]{BDFI.2017}. Mit dem Urteil zur Rechtssache C-362/14, welches auch das "`Schrems-Urteil"' genannt wird, wurde das "`Safe-Harbor"'-Abkommen am 06.10.2015 für ungültig erklärt \autocite[vgl.][]{BDFI.2017}. Zu dem besagten Schrems-Urteil kam der europäische Gerichtshof, da der österreichische Jurist Maximilian Schrems darüber empört war, was das soziale Netzwerk Facebook alles über ihn gespeichert hatte und deshalb gegen Facebook vorgegangen ist. Denn unter anderen hatte Facebook Daten von ihm gespeichert, die Schrems für gelöscht gehalten hatte \autocite[vgl.][]{Welt.2015}. Trotz der Aufhebung des Urteils wird vom US-Handelsministerium weiterhin eine Liste mit den Safe-Harbor-zertifizierten Unternehmen geführt, da die Prinzipien für die Daten, die von den Unternehmen unter der dem Safe Habor-Abkommmen gespeichert wurden, gelten bis sie gelöscht werden \autocite[vgl.][]{BDFI.2017}.
\par
Als Nachfolger von Safe Habor gibt es seit 12.07.2016 das "`Privacy Shield"'-Abkommen zwischen der \ac{EU} und USA. Dieses Abkommen definiert Regelungen, die für ein angemessenes Datenschutzniveau einzuhalten sind. Amerikanische Unternehmen haben die Möglichkeit, sich zur Einhaltung der Privacy Shield-Regelungen zu verpflichten, was die Datenübermittlung von Deutschland nach USA erlaubt \autocite[vgl.][]{LDI.2017}. Das weitverbreitete soziale Netzwerk Twitter hat laut ihren Datenschutzrichtlinien den Regelungen des Privacy Shield-Abkommen verpflichtet, was Twitter den Transfer von deutschen Daten in die USA ermöglicht \autocite[vgl.][]{TwitterInc..2017}.

\subsection{Fall 4: Sonstige Drittstaaten}
Will ein soziales Netzwerk personenbezogene Daten von Deutschland in ein Drittland übermitteln, welches sich keinem der drei zuvor gelisteten Fälle zuordnen lässt, dann muss die übermittelnde Stelle, also der Betreiber des sozialen Netzwerkes, gemäß §4b Abs. 3 und 5 BDSG überprüfen, ob ein angemessenes Datenschutzniveau in bei der Empfängerstelle vorliegt \autocite[vgl.][]{LDI.2017}. Ist diese Überprüfung erfolgreich, so ist die Datenübermittlung legitim. Wenn nicht, dann ist ein Transfer zur Empfängerstelle nur dann möglich, wenn einer der sechs in §4c BDSG beschriebenen Tatbestände eintritt. Der erste Tatbestand in §4c Abs. 1 Nr. 1 BDSG ist derjenige, der als einziger für Betreiber von sozialen Netzwerken greifen kann. Dieser Tatbestand besagt nämlich, dass ein Datentransfer zu einer Empfängerstelle ohne angemessenem Datenschutzniveau dann stattfinden darf, wenn die betroffene Person dazu eingewilligt hat. Damit diese Einwilligung gültig ist, muss die betroffene Person davor ausdrücklich darüber informiert werden, dass ihre Daten ohne angemessenen Datenschutz außerhalb Deutschlands gespeichert oder verarbeitet werden \autocite[vgl.][]{LDI.2017}. Der Kurznachrichtendienst WhatsApp ist ein soziales Netzwerk, deren Datenexport auf dieser Einwilligung beruht. In ihren Datenschutzrichtlinien, die jeder Nutzer akzeptieren muss, heißt es: 
\begin{quotation}
	"`Du akzeptierst unsere Informationspraktiken, [...] sowie die Übertragung und Verarbeitung deiner Informationen in die/den USA und andere/n Länder/n weltweit, [...] und zwar unabhängig davon, wo du unsere Dienste nutzt. Du erkennst an, dass die Gesetze, Vorschriften und Standards des Landes, in dem deine Informationen gespeichert oder verarbeitet werden, von denen deines eigenen Landes abweichen können."' \autocite[][]{WhatsAppInc..2017}
\end{quotation}

\par
Greift keiner der in §4c BDSG beschriebenen Tatbestände, dann gibt es drei letzte Optionen, die ein Datentransfer zu einer Empfängerstelle in einem Drittstaat ohne angemessenen Datenschutzniveau legitim ist:
\begin{enumerate}
	\item Die Empfängerstelle kann einen EU-Standardvertrag unterzeichnen, welcher die Empfängerstelle dazu verpflichtet die Persönlichkeitsrechte der betroffenen Personen zu wahren \autocite[vgl.][]{LDI.2017}. Facebook, eines der bekanntesten sozialen Netzwerken, ist ein Unternehmen, welches in ihren Datenschutzrichtlinien vorgibt, EU-Standardverträge für den Datentransfer aus Europa heraus abgeschlossen zu haben \autocite[vgl.][]{FacebookInc..2017}.
	 
	\item Die Empfängerstelle kann einen Individualvertrag aufsetzen, welcher ebenfalls die Wahrung der Persönlichkeitsrechte der betroffenen garantiert \autocite[vgl.][]{LDI.2017}.
	
	\item Gehört die Empfängerstelle zu einem Konzern, so kann sich dieser Konzern einer verbindlichen Konzernregelung zum Datenschutz verpflichten, welche die Persönlichkeitsrechte der betroffenen Personen wahren. Dies ist besonders für global-aktiven Konzernen interessant, da solche verbindlichen Konzernregelung den Datentransfer zur allen Empfängerstellen des Konzern erlauben, unabhängig davon, in welchem Land sie ansässig sind \autocite[vgl.][]{LDI.2017}.
\end{enumerate}
