\chapter{Grundlagen des Bundesdatenschutzgesetzes}
Im nachfolgenden Kapitel werden die Grundlagen des Bundesdatenschutzgesetzes vermittelt, welche zum allgemeinen Verständnis dieser Arbeit notwendig sind. Im ersten Teil wird der Hintergrund des Gesetzes näher erläutert, um im zweiten Teil auf  die wichtigsten Aspekte einzugehen.
\section{Hintergrund des Gesetzes}
Die amtliche Anmerkung zum Bundesdatenschutzgesetz (\acs{BDSG}) beschreibt den allgemeinen Hintergrund des Gesetzes wie folgt:
\begin{quote}
\glqq Dieses Gesetz dient der Umsetzung der Richtlinie 95/46/EG des Europäischen Parlaments und des Rates vom 24. Oktober 1995 zum Schutz natürlicher Personen bei der Verarbeitung personenbezogener Daten und zum freien Datenverkehr (ABl. EG Nr. L 281 S. 31).\grqq \autocite[][]{DeJureBDSG}
\end{quote}
Die Richtlinie 95/46/EG des Europäischen Parlaments und des Rates vom 24. Oktober 1995 beordert den Schutz natürlicher Personen bei der Verarbeitung personenbezogener Daten. Die Gründe für das Verfassen der Richtlinie sind u.a. einerseits die Notwendigkeit der Übermittlung von personenbezogenen Daten von einem Mitgliedsstaat der Europäischen Union (\acs{EU}) in einen anderen zur Errichtung eines funktionierenden Binnenmarktes. Bei dieser Übermittlung sind jedoch die Grundrechte der Personen zu wahren. Andererseits gelten in den Mitgliedsstaaten unterschiedliche Schutzniveaus der Personenfreiheiten und -rechte bei der Verarbeitung personenbezogener Daten, was die Übermittlung dieser Daten zwischen den verschiedenen Staaten verhindern kann. So könnten zahlreiche gemeinsame Wirtschaftsaktivitäten gehemmt oder der Wettbewerb verfälscht werden. Aus diesem Grund ist ein gleichwertiges Schutzniveau zur Beseitigung der Hemmnisse unerlässlich. In der Gemeinschaft der EU ist die Angleichung der Rechtsvorschriften erforderlich.\autocite[vgl.][]{EU.1995}\\
\section{Grundlegende Aspekte des Gesetzes}
\subsection{Zweck des Gesetzes und Begriffsbestimmungen}
Die Umsetzung der Angleichung ist das BDSG, welches laut \S 1 Abs. 1 BDSG den Einzelnen davor schützt, dass er durch den Umgang mit seinen personenbezogenen Daten in seinem Persönlichkeitsrecht beeinträchtigt wird. \S 1 Abs. 2 BDSG definiert die gültigen Stellen, die personenbezogene Daten für die Erhebung, Verarbeitung und Nutzung erheben:
\begin{enumerate}
\item Öffentliche Stellen des Bundes,
\item Öffentliche Stellen der Länder, soweit der Datenschutz nicht durch Landesgesetz geregelt ist und soweit sie
\begin{itemize}
\item[a)] Bundesrecht ausführen oder
\item[b)] als Organe der Rechtspflege tätig werden und es sich nicht um Verwaltungsangelegenheiten handelt,
\end{itemize}
\item Nicht-öffentliche Stellen (u.a. soziale Netzwerke), soweit sie die Daten unter Einsatz von Datenverarbeitungsanlagen verarbeiten, nutzen oder dafür erheben oder die Daten in oder aus nicht automatisierten Dateien verarbeiten, nutzen oder dafür erheben, es sei denn, die Erhebung, Verarbeitung oder Nutzung der Daten erfolgt ausschließlich für persönliche oder familiäre Tätigkeiten.
\end{enumerate}
Weiterhin grenzt das BDSG den Begriff der personenbezogenen Daten ab. Laut \S 3 Abs. 1 BDSG \glqq sind personenbezogene Daten Einzelangaben über persönliche oder sachliche Verhältnisse einer bestimmten oder bestimmbaren natürlichen Person (Betroffener).\grqq \ \S 3 Abs. 4 BDSG definiert das Verfahren des Speicherns. Das Speichern ist im Einzelnen das \glqq Erfassen, Aufnehmen oder Aufbewahren von personenbezogenen Daten.\grqq \ Diese Daten werden auf einem Datenträger zum Zweck einer Weiterverarbeitung und Nutzung gesichert. Das Übermitteln wird laut \S 3 Abs. 4 BDSG als Bekanntgeben der gespeicherten personenbezogenen Daten an einen Dritten verstanden, wobei die Daten entweder direkt an einen Dritten weitergegeben werden oder dieser die Daten einsieht oder abruft. Werden die gespeicherten Daten letztendlich in jeglicher Weise verwendet, handelt es sich um das Nutzen der Daten.\\
\\Die Erhebung, Verarbeitung und Nutzung personenbezogener Daten steht allgemein unter dem Prinzip der Datensparsamkeit, laut \S 3a TMG. Dies bedeutet, dass die Speicherung der Daten und die Gestaltung der Datenverarbeitungssysteme an dem Ziel ausgerichtet werden, so wenig Daten wie möglich zu erheben, verarbeiten und zu nutzen. Weiterhin sind die Daten zu anonymisieren oder zu pseudonymisieren, sofern dies je nach Verwendungszweck möglich und nicht mit einem erhöhten Aufwand verbunden ist.
\subsection{Zulässigkeit der Datenerhebung}
Zum Abschluss dieses Unterkapitels wird auf einen weiteren grundlegenden Teil des BDSGs eingegangen. Nachdem nun der Zweck des Gesetzes und kontextuelle Begriffe abgegrenzt worden sind, wird nun die Zulässigkeit der Datenerhebung, -verbreitung und -nutzung anhand \S 4 BDSG erläutert.\\
\\\S 4 Abs. 1 BDSG zur Folge ist die \glqq Erhebung, Verarbeitung und Nutzung personenbezogener Daten (...) nur zulässig, soweit dieses Gesetz oder eine andere Rechtsvorschrift dieses erlaubt oder anordnet oder der Betroffene eingewilligt hat.\grqq \ Dies bedeutet, dass die Erhebung, Verarbeitung und Nutzung grundsätzlich verboten ist, jedoch zulässig wird, wenn entweder eine klare Rechtsgrundlage gegeben ist oder der Nutzer die Erhebung, Verarbeitung und Nutzung der Daten ausdrücklich erlaubt.\\
\\Laut \S 4 Abs. 2 BDSG sind personenbezogene Daten beim Betroffenen zu erheben. Dies ist ohne seine Mitwirkung nur zulässig, wenn entweder eine Rechtsvorschrift die Erhebung vorsieht oder eine zu erfüllende Verwaltungsaufgabe oder ein Geschäftszweck diese erforderlich macht. Weiterhin ist sie ebenfalls zulässig, wenn \glqq die Erhebung beim Betroffenen einen unverhältnismäßigen Aufwand erfordern würde\grqq. Generell stehen beide Fälle unter der Bedingung, dass keine Anhaltspunkte für eine Beeinträchtigung überwiegender schutzwürdiger Interessen des Betroffenen existieren.\\
\\\S 4 Abs. 3 BDSG erklärt die Unterrichtungspflicht der Datenerhebung durch die verantwortliche Stelle gegenüber dem Betroffenen. Die Unterrichtung schließt die Identität der verantwortlichen Stelle und den Zweck der Erhebung, Verarbeitung und Nutzung ein. Außerdem sind die Kategorien von Empfängern dem Betroffenen nur mitzuteilen, wenn dieser nicht mit der Übermittlung an diese zu rechnen hat. Werden personenbezogene Daten durch die Anordnung einer Rechtsvorschrift erhoben, die zu einer Auskunft verpflichtet, ist der Betroffene hierauf hinzuweisen.\\
\\Nachdem nun die Grundlagen des BDSGs erläutert worden sind, folgt im nachfolgenden Kapitel die Beschreibung der Grundsätze des Telemediengesetzes, welches die Vorschriften des BDSGs auf die Erhebung personenbezogener Daten bei der Benutzung des Internets und folglich sozialer Netzwerke anwendet.