\chapter{Grundlagen des Bundesdatenschutzgesetzes}
Im nachfolgenden Kapitel werden die grundlegenden Aspekte des Bundesdatenschutzgesetzes vermittelt, welche zum allgemeinen Verständnis dieser Arbeit notwendig sind. Im ersten Teil wird der Hintergrund des Gesetzes näher erläutert, um im zweiten Teil die wichtigsten Texte zu erläutern.
\section{Hintergrund des Gesetzes}
Die amtliche Anmerkung zum Bundesdatenschutzgesetz (\acs{BDSG}) beschreibt den allgemeinen Hintergrund des Gesetzes wie folgt:\\
\\\glqq Dieses Gesetz dient der Umsetzung der Richtlinie 95/46/EG des Europäischen Parlaments und des Rates vom 24. Oktober 1995 zum Schutz natürlicher Personen bei der Verarbeitung personenbezogener Daten und zum freien Datenverkehr (ABl. EG Nr. L 281 S. 31).\grqq \autocite[][]{DeJureBDSG} \\
\\Die Richtlinie 95/46/EG des Europäischen Parlaments und des Rates vom 24. Oktober 1995 beordert den Schutz natürlicher Personen bei der Verarbeitung personenbezogener Daten. Die Gründe für das Verfassen der Richtlinie sind u.a. einerseits die Notwendigkeit der Übermittlung von personenbezogenen Daten von einem Mitgliedsstaat der Europäischen Union (\acs{EU}) in den anderen zur Errichtung eines funktionierenden Binnenmarktes. Bei dieser Übermittlung sind jedoch die Grundrechte der Personen zu wahren. Andererseits gelten in den Mitgliedsstaaten unterschiedliche Schutzniveaus der Personenfreiheiten und -rechte bei der Verarbeitung personenbezogener Daten, was die Übermittlung dieser Daten zwischen den verschiedenen Staaten verhindern kann. So könnten zahlreiche gemeinsame Wirtschaftsaktivitäten gehemmt oder der Wettbewerb verfälscht werden. Aus diesem Grund ist ein gleichwertiges Schutzniveau zur Beseitigung der Hemmnisse unerlässlich. In der Gemeinschaft ist die Angleichung der Rechtsvorschriften erforderlich.\autocite[vgl.][]{EU.1995}\\
\\