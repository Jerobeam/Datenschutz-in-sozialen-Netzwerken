\chapter{Grundlagen des Telemediengesetzes im Bezug auf soziale Netzwerke}
Das BDSG regelt den Datenschutz bei der allgemeinen Erhebung von personenbezogenen Daten, jedoch nicht explizit für die Datenerhebung bei der Benutzung des Internets und folglich sozialer Netzwerke. Für diesen Zweck ist 2007 das \ac{TMG} in Kraft getreten, welches den Datenschutz im Internet regelt.\autocite[vgl.][]{klicksafe.de} \autocite[vgl.][]{shr.de} In diesem Kapitel werden die Grundlagen des TMGs vermittelt und im Kontext der Anwendung von sozialen Netzwerken behandelt.
\section{Begriffsbestimmungen und Pflichten}
Laut \S 2 TMG ist ein Diensteanbieter \glqq jede natürliche oder juristische Person, die eigene oder fremde Telemedien zur Nutzung bereithält oder den Zugang zur Nutzung vermittelt (...).\grqq \ Weiterhin ist ein Nutzer laut \S 2 TMG jede natürlich oder juristische Person, die Telemedien nutzt, um insbesondere Informationen zu erhalten. Abschließend werden Telemedien als Verteildienste bezeichnet, welche laut \S 2 TMG \glqq im Wege einer Übertragung von Daten ohne individuelle Anforderung gleichzeitig für eine unbegrenzte Anzahl von Nutzern erbracht werden.\grqq \ Dazu gehören nahezu alle Angebote im Internet, wie Shopping-Portale, Online-Dienste (wie z.B. Wetter- oder Nachrichtenauskünfte) oder Suchmaschinen.\autocite[vgl.][]{shr.de}\\
\\Wann und unter welchen Bedingungen personenbezogene Daten nun aber tatsächlich erhoben werden dürfen, regeln \S\S 12, 13 TMG.  \S 12 Abs. 1 TMG besagt, dass Diensteanbieter personenbezogene Daten nur erheben und verwenden dürfen, wenn das TMG oder eine andere Telemedien-bezügliche Rechtsvorschrift es erlaubt oder der Nutzer eingewilligt hat. Gleiches gilt für die Verwendung der Daten für andere Zwecke. Zu Beginn des Nutzungsvorgangs hat der Diensteanbieter den Betroffenen über die Erhebung und Verwendung der Daten laut \S 13 Abs. 1 TMG zu unterrichten. Die Einwilligung des Nutzers kann dabei elektronisch eingeholt werden, sofern der Diensteanbieter sicherstellt, dass die Einwilligung bewusst und eindeutig erteilt wird, die Einwilligung protokolliert wird, der Inhalt der Einwilligung jederzeit abrufbar ist und der Nutzer die Einwilligung jederzeit widerrufen kann (\S 13 Abs. 2 TMG).
\section{Speicherung von Bestands- und Nutzungsdaten}
\S 14 Abs. 1 TMG spezifiziert eine Art der Daten, welche von den Diensteanbietern erhoben und verwendet werden dürfen: Die Bestandsdaten. Bestandsdaten gelten als Daten, welche \glqq für die Begründung, inhaltliche Ausgestaltung oder Änderung eines Vertragsverhältnisses zwischen dem Diensteanbieter und dem Nutzer über die Nutzung von Telemedien erforderlich sind (Bestandsdaten).\grqq \ Wenn eine zuständige Stelle eine Auskunft über Bestandsdaten anordnet, sofern dies staatliche Zwecke, wie z.B. der Strafverfolgung, notwendig ist, dürfen Diensteanbieter diese Auskunft erteilen.\\
\\Neben den Bestandsdaten sieht das TMG ebenfalls die Erhebung und Nutzung von Nutzungsdaten vor. Laut \S 15 Abs. 1 TMG sind Nutzungsdaten erforderlich, um die Inanspruchnahme von Telemedien zu ermöglichen und abzurechnen. Dabei sind Nutzungsdaten insbesondere Merkmale zur Identifikation des Nutzers, Angaben zum Beginn und Ende sowie zum Umfang der jeweiligen Nutzung und Angaben über die vom Nutzer in Anspruch genommenen Telemedien. \S 15 Abs. 3 erlaubt den Dienstanbietern die Nutzung der Daten zur Erstellung von Nutzungsprofilen für Zwecke der Werbung, Marktforschung oder zur bedarfsgerechten Gestaltung der Telemedien führen, wenn der Nutzer dem nicht widerspricht. Weiterhin darf der Dienstanbieter die Daten auch nach dem Ende des Nutzungsvorgangs für Zwecke der Abrechnung verwenden und diese für demselben Zweck an andere Dienstanbieter und Dritte weitergeben. Zur Marktforschung anderer Dienstanbieter dürfen die Daten anonymisiert übermittelt werden (\S 15 Abs. 4,5 TMG).\\
\section{Die Gesetze zum Datenschutz im Kontext sozialer Netzwerke}
\label{KontextSozialeNetzwerke}
In der Praxis sind die Dienstanbieter von sozialen Netzwerken darauf angewiesen, mit ihrem Angebot Profit zu erzielen. Dabei werden die verwendeten Daten, wie weiter oben erläutert, häufig anonymisiert an Dritte verkauft und weitergegeben, um die grundsätzliche kostenlose Nutzung ihrer Dienste auszugleichen. Der Verkauf der Daten wird über die Allgemeinen Geschäftsbedingungen (\acs{AGB}) abgesichert, welche vor der Nutzung vom Benutzer durchgelesen werden sollten und bestätigt werden. Somit ist die Unterrichtungspflicht der Diensteanbieter gegenüber den Nutzern erfüllt. Jedoch ist nicht alles, was die Anbieter in den AGB vermerken, automatisch rechtskräftig. Aus diesem Grund geht der Bundesverband der Verbraucherzentralen gegen einzelne, für Nutzer besonders nachteilige Klauseln vor.\autocite[vgl.][]{klicksafe.de2}\\
\\Im Jahr 2015 beispielsweise hat der Verbraucherschutz \textit{Facebook} abgemahnt, nachdem der Anbieter des sozialen Netzwerks seine AGBs aktualisiert hat und diese aus Sicht des Verbands gegen 19 Klauseln des deutschen Rechts verstoßen. Das Geschäftsmodell von Facebook basiert auf dem Motto \textit{Facebook ist und bleibt kostenlos}, jedoch wurden mit der Einführung der aktualisierten AGB personenbezogene Nutzerdaten an werbetreibende Unternehmen weiterverkauft. Die Verbraucherzentrale argumentiert, dass Facebook sein Geschäftsmodell verharmlosen und Transparenz verhindern würde. Trotz Mahnung der Verbraucherzentrale konnte noch keine Einigung mit Facebook erzielt werden, sodass den Nutzern aktuell (Stand April 2017) nichts anderes übrig bleibt, als das soziale Netzwerk zu verlassen.\autocite[vgl.][]{SpiegelOnline} \autocite[vgl.][]{Verbraucherzentrale}\\
\\Weiterhin gilt für soziale Netzwerke ebenfalls das Prinzip der Datensparsamkeit. Dies bedeutet, dass auch soziale Netzwerke die Datenmenge, die der Nutzer dem Netzwerk übermittelt, auf ein Minimum zu reduzieren hat. Außerdem ist es in einzelnen Fällen für den Nutzer möglich, sich unter einem Pseudonym anzumelden und selektiv mit den Daten umzugehen: "`bei den wenigsten Sozialen Netzwerken ist es wirklich notwendig, seinen vollen Namen, die echte Adresse oder die Telefonnummer anzugeben, um den Dienst nutzen zu können. Schließlich kauft man dort (...) in der Regel nicht ein oder erhält Rechnungen, wofür der Anbieter Geschäftsdaten benötigen würde."' \textsuperscript{ }\autocite[][]{klicksafe.de2}\textsuperscript{,} \autocite[vgl.][]{klicksafe.de2}\\
\\An dieser Stelle sind die nötigen Grundlagen zum BDSG und TMG vermittelt. Im folgenden Kapitel werden internationale Datenschutzregelungen im Kontext der Verwendung von sozialen Netzwerken näher beleuchtet.